\documentclass[a4paper]{article}
\usepackage{amssymb,amsmath}
\usepackage[utf8]{inputenc} 
\usepackage[T1]{fontenc} 
\usepackage{xcolor}
\usepackage[normalem]{ulem}

\usepackage[noadjust]{cite}

\newcommand{\alexei}[1]{{\color{blue}#1}}
\newcommand{\barbara}[1]{{\color{blue}#1}}
\newcommand{\bakur}[1]{{\color{blue}#1}}
\newcommand{\saman}[1]{{\color{magenta}#1}}
\newcommand{\lpg}[1]{{\color{red}#1}}
\newcommand{\comment}[1]{{\color{blue}#1}}

\newcommand{\bp}[1]{{\color[rgb]{0.43,0.21,0.1}#1}}
\newcommand{\ps}[1]{{\color[rgb]{0.1,0.5,0}#1}}
\bibliographystyle{JHEP}

%============= FORMATING (A4) ======================================
  \setlength{\textwidth}{16cm}
  \setlength{\textheight}{24.5cm}
  \setlength{\oddsidemargin}{-0.2
  cm}
  \setlength{\topmargin}{-1.7cm}
%\def\baselinestretch{2}

\begin{document} 

\section*{Reply to the Editor}

\ \\
Dear Editor,

\ \\
On July 9, 2020 we were  notified of the Editorial Decision
that our manuscript JHEP 157P 0620 would no longer be
considered for publication on the basis of the Referee's Report of our manuscript.
Upon a careful inspection of the Report we agree 
with the Referee's overall assessment on how to
improve our work by carrying out our study in a rigorous TMD factorization framework.
Therefore on the basis of the Referee's constructive criticism 
we have revised our manuscript accordingly, addressing all issues raised 
in the referee's report.

\ \\
We feel that incorporating the Referee's recommendations 
has significantly improved our work. For instance, in the 
revised manuscript the CSS evolution is handled rigorously 
and according to current state-of-the-art standards.

\ \\
We therefore would like to appeal the Editor's decision and kindly request 
a second review of our significantly revised manuscript.

\ \\
We will be happy if the same Referee or a new Referee is asked to review 
the revised manuscript --- which ever the Editor deems appropriate.

\ \\
Sincerely,

\ \\
Saman Bastami, Leonard Gamberg, Bakur Parsamyan, \\
Barbara Pasquini, Alexei Prokudin and Peter Schweitzer

\newpage
\section*{Reply to the Referee}

\ \\
We would like to thank the Referee for the careful review
and constructive criticism. We agree in all cases that the 
points raised by the Referee needed to be addressed. Following
the Referee's recommendations, our manuscript went through 
major revisions which, in our opinion, have significantly improved 
our work. In the following we respond to the points raised by 
the Referee and outline our revisions.

\ \\
The Referee has pointed out in the Report that important contributions
made in Refs.~\cite{Gehrmann:2014yya,Echevarria:2015byo,Echevarria:2015usa,Echevarria:2016scs,Li:2016ctv,Vladimirov:2016dll,Luo:2019hmp,Luo:2019szz,Ebert:2020yqt,Gutierrez-Reyes:2017glx,Gutierrez-Reyes:2018iod,Ji:2006ub,Koike:2007dg,Sun:2013hua,Dai:2014ala,Scimemi:2019gge,Collins:1981va,Aybat:2011zv,Angeles-Martinez:2015sea,Ebert:2020dfc} 
were not cited in the previous version of the manuscript. 
We regret that the bibliography did not reflect the significant developments in the field in recent years.
We remedied this point and included in the revised manuscript 
all references pointed out by the Referee. In this context we also added 
the recent work \cite{Moos:2020wvd} which appeared after the Referee Report
was written (and where important results for the pretzelosity distribution
were derived). The exact contexts where
the new references are cited are described in the Summary of Changes. 

\ \\
We agree with the Referee that our previous discussion of TMD 
evolution was inadequate. 
The Referee rightfully criticized that in Eqs.~(2.4)-(2.6) we used
an inappropriate notation without clearly indicating the scales 
$(\mu,\,\zeta)$. The Referee wrote: 
``I cannot accept the work if at least a check
with the CSS evolution is done.''
In the revised manuscript all these points are addressed:   
the Eqs.~(2.4)-(2.6) have been modified, and the Sec.~2.5 is 
entirely rewritten. 
We not only run a ``check with the CSS evolution,'' but completely
changed our approach, by implementing Next-to-Leading Logarithm (NLL) CSS evolution in $b_T$-space,
and re-calculating all numerical results. The new results confirmed
our previous estimates based on the Gaussian model, especially if 
one considers the sizeable error bars of the current data. However,
at the high COMPASS energies with $\langle Q^2\rangle=28\,{\rm GeV}^2$
perturbative evolution effects play an important role as the Referee
correctly pointed out. With the re-calculated results we feel
more confident when confronting our predictions with the data.

\ \\
The Referee's question ``from where to where [we are] 
considering the TMD evolution'' shows that we failed to explain this
important point which is central for our work. We added several paragraphs 
in Sec.~1 and Sec.~2.5 to explain this point which we would
like to re-iterate also in  this Response. 

\ \\
The challenges we deal with in our work are as follows. 
We use quark models to estimate the nonperturbative hadronic 
matrix elements which define TMDs. Quark models are of eminent importance 
for hadron spectroscopy; they provide a great deal of intuition about hadrons,
successfully explain many (though not all) hadronic properties, but 
also have their limitations and the precise connection to QCD is not 
fully clarified. The specific question 
we wish to address in our work is whether quark models can catch some of 
the main features of pion and nucleon structure. When using quark models 
we deal with 2 nontrivial challenges.

\ \\
First, is it justified to describe the proton as 3 ``valence $q$'' 
and the pion as 1 ``valence $q$'' and 1 ``valence $\bar{q}$"? If so, 
then such a quark model picture is expected to apply at a very low 
hadronic scale $\mu_0 \lesssim {\cal O}(M_{\rm had})$ where $M_{\rm had}$ 
is of the order of magnitude of the nucleon mass. The 
underlying assumption is that at such a low scales the QCD quark degrees 
of freedom can be approximated by effective constituent-quark or 
quark-diquark degrees of freedom, depending on the model.  
In order to address this aspect, one is motivated to use models to evaluate the nonperturbative matrix elements and apply the predictions to phenomenology. 
Prior studies have indicated 
that quark models can give useful description of azimuthal and spin
asymmetries in SIDIS within a model accuracy of (20--30)$\,\%$  
in the valence-$x$ region $x\gtrsim 0.1$. It is important to explore
complementary processes, such as Drell-Yan, to further test the models 
and acquire a more complete picture of their applicability. This is one 
main motivation for our work.

\ \\
Second, how can one apply model results calculated at low hadronic scales 
to the description of high energy experiments like Drell-Yan? For that it 
is indispensable to apply TMD evolution. Let us first discuss the simpler
case of PDFs characterized by a single scale, the renormalization scale
$\mu$. The initial quark model scale $\mu_0$ can be defined as follows. 
The fraction 
$M_2^{\rm val}(\mu)=\sum_q\int_0^1dx\,x(f_1^q-f_1^{\bar q})(x,\mu)$
of the nucleon momentum 
carried by valence quarks at experimental scales $\mu$ is known 
from PDF parametrizations. This fraction increases at lower $\mu$
under DGLAP-evolution. The initial quark model 
scale is defined as that scale $\mu_0$ where $M_2^{\rm val}(\mu_0)=1$ holds
\cite{Traini:1997jz}. This method yields $\mu_0 \sim 0.5\,{\rm GeV}$
for both pion and nucleon with the exact value depending on whether 
one uses LO or NLO DGLAP evolution.

\ \\
At this point we wish to clarify a misunderstanding. The Referee seems 
to have the impression that $\mu_0\ll 1\,{\rm GeV}$. This would indeed 
make quark model applications questionable. The initial scale $\mu_0$ 
is below $1\,{\rm GeV}$ but clearly above the Landau pole. It is important 
to stress that we do not claim that perturbative QCD generally works in 
this region. However, DGLAP evolution does, in the following sense:
the works by the GRV (and GRS) groups on parametrizations of nucleon 
(and pion)
PDFs show a remarkable perturbative stability between LO and NLO fits
indicating applicability of DGLAP at initial scales as low as 
$\mu_0^2 = 0.26\,{\rm GeV}^2$ 
\cite{Gluck:1991ng,Gluck:1991ey,Gluck:1994uf,Gluck:1998xa,Gluck:1999xe}.
This indicates that quark models can be used to
provide, in a first approximation, nonperturbative information on PDFs.
Admittedly, the GRV parametrizations not only have valence quarks at 
their initial scales, but also sea-quark and gluon distributions.
But those are much smaller than the valence quark distributions and in
this sense constitute a ``small correction'' of about (20--30)$\,\%$ 
to the leading-order description of hadrons in the Fock expansion.


\ \\
For TMDs the situation is more complex as they depend on a second 
scale $\zeta$. When applying TMD evolution we face 2 open questions: 
(i) no rigorous criterion is known how to determine the initial 
scale $\zeta_0$ in quark models, though it may be natural to assume 
$\zeta_0 \sim \mu_0^2$; and 
(ii) there is no expertise with CSS evolution at low scales
(in contrast to the positive GRV-experience with
DGLAP evolution).

\ \\
The ultimate goal is to implement TMD evolution from 
low initial quark model scales $(\mu_0,\,\zeta_0)$ to experimentally 
relevant scales. But because of the above-described challenges this 
is not possible at present. We therefore proceed as follows. 
In the revised manuscript 
we take the model PDFs or relevant transverse moments of TMDs and evolve 
them from their initial scales to $\mu^2 = 2.5\,{\rm GeV}^2$ using DGLAP 
(this step is supported by the GRV experience).
We employ exact DGLAP evolution where possible or ``approximate DGLAP 
evolution,'' e.g.\ in the cases like the Boer-Mulders functions for
which we use the DGLAP evolution of the chiral-odd transversity PDF, 
an approximation often used in literature. In this way we evolve
the model PDFs or transverse moments from the low initial scale $\mu_0^2$ 
to $\mu^2 = 2.5\,{\rm GeV}^2$ where phenomenological information on many 
TMDs is available from parametrizations. From  
$\mu^2 = 2.5\,{\rm GeV}^2$ and $\zeta = 2.5\,{\rm GeV}^2$ 
we then continue the evolution of the model results up to COMPASS scales
by applying NLL CSS evolution whose parameters are constrained by 
phenomenological studies and TMD fits.

\ \\
This allows us to test the $x_N$-dependence 
(determined by the models of the nucleon structure)
and the $x_\pi$-dependence (dictated by the models of the pion structure) 
of model TMDs. At this point we cannot test 
the  $k_T$-dependence of model TMDs. This will be addressed in future
studies. We feel that our new results, based on a state-of-the-art CSS
evolution, are more reliable and can be compared to the COMPASS data
with more confidence. The experimental uncertainties 
are still large. Our results are nevertheless much appreciated by the 
experimental colleagues, especially in the case of azimuthal Drell-Yan
single-spin asymmetries for which only few or no calculations are available.

\ \\
The Referee also raised the concern whether it is legitimate to apply 
the TMD formalism to the description of COMPASS data considering
that the covered range of transverse dilepton momenta is 
$0.4 < q_T < 5\,{\rm GeV}$ with the upper limit of the $q_T$ region 
of the order of the hard scale $Q$. Indeed, the tail of the $q_T\sim Q$ cross section
should be described by the fixed-order (FO) contribution to the cross section within the collinear factorization framework, and a matching of the TMD and FO contributions should be carried out. Thus, we share the Referee's concern. 
However, we have run  some tests and we find that the majority of the COMPASS data
is in the TMD region. From experiment,  it is known, 
 that the average $\langle q_T^2/Q^2\rangle \lesssim 0.1$. Further  we verified that our numerical results
(for asymmetries) have little sensitivity to whether one uses 
3, 4, or 5 GeV as an upper bound for $q_T$-integration 
(total cross sections might be more sensitive).
We are therefore confident that the application of the TMD formalism
is justified to a good approximation, especially if one considers
the error bars of the current data. But the Referee  is correct,  and we added a note of caution that this point may need
a revision when dealing in future with more precise data.

\ \\
Overall we believe that we have considerably improved our manuscript.
We refer to the Summary of Changes for a detailed discussion of 
all our revisions.

%\newpage
\section*{Summary of Changes}

In the following we describe the changes made in the manuscript in
the order of their occurrence.

\begin{itemize}

\item\comment{
Abstract, the following sentence has been added:}

TMD evolution is implemented at Next-to-Leading Logarithmic precision for
the first time for all asymmetries. 


\item\comment{
Sec.~I ``Introduction.'' The penultimate paragraph is replaced by the
following text. Hereby the Refs.~\cite{Gehrmann:2014yya,Echevarria:2015byo,Echevarria:2015usa,Echevarria:2016scs,Li:2016ctv,Vladimirov:2016dll,Luo:2019hmp,Luo:2019szz,Ebert:2020yqt,Gutierrez-Reyes:2017glx,Gutierrez-Reyes:2018iod,Ji:2006ub,Koike:2007dg,Sun:2013hua,Dai:2014ala,Scimemi:2019gge,Collins:1981va,Aybat:2011zv,Angeles-Martinez:2015sea,Ebert:2020dfc} 
pointed out by the Referee as well as the preprint \cite{Moos:2020wvd} 
(which appeared after the Referee Report was prepared) have been added.
Notice that the numbering of the references in this Reply (here and in the
following) differs from the numbering of the references in the manuscript.}

One key aspect in our study is the evolution of model results from 
the low hadronic scales to experimentally relevant scales. For that 
(i) knowledge of the low initial scale, and (ii) applicability of evolution 
equations at low scales are crucial. Both requirements are fulfilled in
the case of parton distribution functions which depend on one scale only, 
the renormalization scale~$\mu$.
First, the value of the initial quark model scale $\mu_0$ can be consistently
determined by evolving the fraction of nucleon momentum carried by valence 
quarks, $M_2^{\rm val}(\mu) = \int dx\,x(f_1^q-f_1^{\bar q})(x,\mu)$,
known from parametrizations, using DGLAP evolution down to that scale
$\mu_0$ at which valence quarks carry the entire nucleon momentum, 
i.e.\ $M_2^{\rm val}(\mu_0) = 1$ \cite{Traini:1997jz}. 
Numerically it is $\mu_0\sim 0.5\,{\rm GeV}$. %\bakur{(0.5\,{\rm GeV}?)}. 
Second, work by the GRV and GRS groups on parametrizations of nucleon and 
pion unpolarized parton distribution functions show remarkable perturbative
stability between LO and NLO fits indicating applicability of DGLAP 
evolution down to initial scales as low as $\mu_0^2 = 0.26\,{\rm GeV}^2$ 
% thanks to the reasonably small value of the expansion parameter 
% $\frac{\alpha_a(\mu_0)}{4\pi} \sim 0.1$
\cite{Gluck:1991ng,Gluck:1991ey,Gluck:1994uf,Gluck:1998xa,Gluck:1999xe}.

% TMD evolution \cite{Collins:2011zzd} plays an important role 
% in interpreting the data and understanding of TMDs.
TMDs depend not only on the renormalization scale $\mu$ but also on 
the rapidity scale $\zeta$ \cite{Collins:2011zzd}. The theoretical and
phenomenological understanding of TMDs witnessed an incredible rate of 
developments in the recent years including NNLO and NNNLO calculations 
of the evolution kernel of unpolarized TMDs
\cite{Gehrmann:2014yya,Echevarria:2015byo,Echevarria:2015usa,Echevarria:2016scs,Li:2016ctv,Vladimirov:2016dll,Luo:2019hmp,Luo:2019szz,Ebert:2020yqt},
NLO calculations for the quark helicity distribution 
\cite{Gutierrez-Reyes:2017glx}, 
NLO \cite{Gutierrez-Reyes:2017glx}, and NNLO \cite{Gutierrez-Reyes:2018iod},
calculations for transversity and pretzelosity, 
and NLO calculations for the Sivers function 
\cite{Ji:2006ub,Koike:2007dg,Sun:2013hua,Dai:2014ala,Scimemi:2019gge}.
Recently also the first non-trivial expression for the small-$b$
expansion of the pretzelosity distribution was derived \cite{Moos:2020wvd}.
However, in the context of quark model applications we face two challenges.
First, no rigorous (analog to the $\mu_0$-determination) criterion exists 
to fix the value of the initial rapidity scale $\zeta_0$ of quark models, 
though an educated guess may be $\zeta_0\sim\mu_0^2$.
Secondly, in the case of Collins-Soper-Sterman (CSS) evolution~\cite{Collins:1984kg}, 
% \sout{CSS evolution}
 no expertise is available analog 
to the GRV/GRS applications of DGLAP evolution starting from low hadronic scales.

In this situation in previous quark model studies 
TMD evolution effects were often estimated approximately
\cite{Boffi:2009sh,Pasquini:2011tk,Pasquini:2014ppa} based on
an heuristic Gaussian Ansatz for transverse parton momenta with
energy dependent Gaussian widths. While providing a useful 
description of data on many processes including pion-induced Drell-Yan
\cite{Schweitzer:2010tt}, it is important to improve the simple Gaussian treatment in view of the recent progress in the TMD theory  
\cite{Gehrmann:2014yya,Echevarria:2015byo,Echevarria:2015usa,Echevarria:2016scs,Li:2016ctv,Vladimirov:2016dll,Luo:2019hmp,Luo:2019szz,Ebert:2020yqt,Gutierrez-Reyes:2017glx,Gutierrez-Reyes:2018iod,Ji:2006ub,Koike:2007dg,Sun:2013hua,Dai:2014ala,Scimemi:2019gge,Moos:2020wvd}.
We will therefore use TMD evolution~\cite{Collins:2011zzd} 
at  Next-to-Leading Logarithm (NLL) precision 
to describe the transverse momentum dependence of the Drell-Yan process.
As presently application of TMD evolution at the low quark 
model scales below 1 GeV is not known and we shall proceed in two steps. We will evolve adequately weighted
transverse moments of TMDs from the low initial scale $\mu_{0}^{2}$ to a scale of $Q_0^2=2.4\,{\rm GeV}^2$
at which phenomenological information on transverse momentum dependence 
is available from TMD fits~\cite{Su:2014wpa, Kang:2014zza,Kang:2015msa, Bacchetta:2017gcc,Scimemi:2019cmh,Bacchetta:2019sam} of polarized and unpolarized SIDIS, DY and weak boson productions data, and use NLL TMD  evolution to evolve to the scales relevant in the % COMPASS experiment 
COMPASS Drell-Yan measurements, i.e., $\langle Q^2\rangle=28\,{\rm GeV}^2$.
In this way we will be able to test the $x$-dependencies of the model TMDs 
while the description of the $q_T$ dependencies of the DY observables
is described on the basis of TMD fits. 
%In order to test also the transverse 
%parton momentum dependence of the TMDs predicted in models, it is necessary
%to implement CSS evolution at very low scales and demonstrate its applicability.
%These aspects will be presented in future studies.
%
%In Refs.~\cite{Wang:2017zym,Vladimirov:2019bfa} TMD evolution was shown 
%to be important for the description of the pion-induced unpolarized DY 
%cross-sections. TMD evolution has been studied for the 
%COMPASS DY data and asymmetries also in
%Refs.~\cite{Vladimirov:2019bfa,Li:2019uhj,Ceccopieri:2018nop,
%Wang:2018naw,Wang:2018pmx}. 
%QCD evolution is particularly important when applying 
%CQM results obtained at a low initial scale to 
%the description of high-energy processes. 
%Given the precision of current data and accuracy of models,
%we will content ourselves with an approximate evolution method 
%shown to provide good estimates of evolution effects in prior studies.
%
For completeness we remark that the importance of TMD evolution
for the description of pion-induced DY and the recent COMPASS
data was also studied in
Refs.~\cite{Wang:2017zym,Vladimirov:2019bfa,Li:2019uhj,Ceccopieri:2018nop,
Wang:2018naw,Wang:2018pmx}.


\item\comment{
Sec.~2.1, after Eqs.~(2.1)-(2.2) the clarifying sentence is added:}

The TMDs depend on renormalization and rapidity scales which are not
indicated for brevity in (2.1) and (2.2) and will be discussed in Sec.~2.2.


\item\comment{
The Secs.~2.2 and 2.3 are completely rewritten,
supersede the Secs.~2.2 and 2.5 of the previous version, and provide a detailed
description of how the CSS evolution is implemented. The changes are too
extensive to be copied and pasted here, and we refer to the manuscript. 
The Refs.~\cite{Collins:1981va,Aybat:2011zv,Angeles-Martinez:2015sea,
Scimemi:2019cmh,Bacchetta:2019sam,Ebert:2020dfc} pointed out by the 
Referee are cited in the second paragraph of Sec.~2.2.}


\item\comment{The following paragraphs are added in sec.~2.5 (sec.~2.4 of the previous version) in accordance with the revisited evolution mechanism:}

{Presently there is no experience in literature with implementing
CSS evolution at scales as low as $\mu_0\sim 0.5\,{\rm GeV}$
to which the quark model results refer. Therefore the scale
$Q_0^2 = 2.4\,{\rm GeV}^2$ was chosen in Sec.~2.3 
as the initial scale for the CSS evolution.
The evolution effects between $\mu_0$ and $Q_0$ cannot be determined 
exactly in the CSS formalism, and they also cannot be neglected. 
We therefore estimate them as follows.  
We start with the model predictions for the parton
distributions or transverse moments of TMDs as they 
appear in Eq.~(2.25) at the initial quark model 
scale $\mu_0$. We evolve them using LO DGLAP evolution to 
the scale $Q_0$. In contrast to CSS, experience with 
implementing DGLAP evolution at low scales is available \cite{Gluck:1991ng,Gluck:1991ey,Gluck:1994uf,Gluck:1998xa,Gluck:1999xe}. 
Hereby we use exact DGLAP evolution for $f_{1,h}^a(x)$ and $h_{1,p}^a(x)$.
In all other cases we use approximate DGLAP evolution: for the transverse
momenta of the proton Sivers function we use the $f_{1,h}^a(x)$-nonsinglet 
evolution shown to lead good results in the LFCQM model study of SIDIS 
asymmetries \cite{Pasquini:2011tk}, while all chiral-odd TMDs we assume 
the DGLAP evolution of transversity \cite{Hirai:1997mm,Boffi:2009sh}.
For the $k_T$-dependencies of the TMDs we use the same
input from TMD parametrizations as described in Eq.~(2.25).

The predictions from both models evolved in this way 
are shown along with the available parametrizations in
Figs.~3-4 at the scale $Q_0$. 
It is important to stress that in this way we are able to test the
$x$-dependencies of the model predictions against the COMPASS data. 
The ultimate goal would be to test similarly also the quark model 
predictions for $k_T$-dependencies. This requires an implementation of 
the CSS evolution starting from low initial scales $\mu_0< 1\,{\rm GeV}$
which is beyond the scope of this work, and will be addressed in future studies.} 


\item\comment{Figs.~3 and 4 are now presenting the input functions from models and phenomenological fits at the scale $Q_0^2=2.4$~GeV$^2$} 

\item\comment{The Figs.~5, 6, 7, 8, 9 are replaced with new 
figures showing the new results obtained with NLL CSS evolution.}


\item\comment{Sec.~3.3, the following paragraph is added at the 
end of this section:}

Before ending this section it is important to remark that the COMPASS
experiment has covered the range $0.4\,{\rm GeV} < q_T < 5\,{\rm GeV}$. 
At the upper limit the condition $q_T \ll  Q$ for
the applicability of the TMD factorization is not satisfied which 
constitutes an  uncertainty in our calculations.
However, in the experiment (and in our calculation) it is 
$\langle q_T\rangle = \bakur{1.2} \,{\rm GeV}$ which is much smaller 
than $\langle Q \rangle = 5.3\,{\rm GeV}$ and we verified that 
the region of large $q_T$ (namely, $3\,{\rm GeV} < q_T < 5\,{\rm GeV}$) 
in our calculations has a negligible impact on
the $q_T$-averaged (integrated) asymmetries in the experiment.
Considering the accuracy of models, we believe that this is not 
the dominant uncertainty in our calculation. However, in future 
when more precise data will become available this point will have
to be carefully revisited.


\item\comment{Sec.~4, Conclusions. The first sentence of the
second paragraph is modified as follows:}

We presented a complete description of polarized DY at leading twist using TMD evolution at NLL accuracy. 


\item\comment{Bibliography. 
The following references were added during the revisions:}
 Refs.~\cite{Gehrmann:2014yya,Echevarria:2015byo,Echevarria:2015usa,Echevarria:2016scs,Li:2016ctv,Vladimirov:2016dll,Luo:2019hmp,Luo:2019szz,Ebert:2020yqt,Gutierrez-Reyes:2017glx,Gutierrez-Reyes:2018iod,Ji:2006ub,Koike:2007dg,Sun:2013hua,Dai:2014ala,Scimemi:2019gge,Collins:1981va,Aybat:2011zv,Angeles-Martinez:2015sea,Ebert:2020dfc,Scimemi:2018xaf,DAlesio:2004eso,Efremov:2010mt,Schweitzer:2012hh,Boer:2011xd,Bacchetta:2019qkv,Qiu:2000ga,Moch:2005id,Kang:2011mr,Echevarria:2012pw,Grozin:2014hna,Collins:2017oxh,Pasquini:2007iz,Traini:1997jz,Moos:2020wvd,Su:2014wpa,Kang:2015msa,Kang:2014zza,Bacchetta:2017gcc}.

% PS: INTERNAL REMARK TO OURSELVES.
% Actually \cite{Scimemi:2019cmh} was mentioned in our preprint.
% We wrote: `` In order to do it in the full complexity one needs to know 
% details of the evolution kernel $S$ in all values of $b_{T}$, which is 
% currently under debate, see for instance Ref.~\cite{Scimemi:2019cmh}.''
% Also \cite{Bacchetta:2019sam} was mentioned before. Both were preprints 
% and are meanwhile published. I updated them in the bib-file. \saman{There other references added to the bib, should we address them as well here?}\\
% \ps{What are the "other references" added in the bib?\\
% The above "internal comment" can be deleted.}

\end{itemize}

\newpage

\bibliography{biblio_DY} 

\end{document}

